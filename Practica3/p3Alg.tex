

% microtype: Tipografía.
% mathpazo: Usa la fuente Palatino.
\documentclass[a4paper, 11pt]{article}
\usepackage[protrusion=true,expansion=true]{microtype}
\usepackage{mathpazo}
\usepackage{booktabs}
\usepackage{multicol}
\usepackage{multirow}

% Indentación de párrafos para Palatino
\setlength{\parindent}{0pt}
  \parskip=8pt
\linespread{1.05} % Change line spacing here, Palatino benefits from a slight increase by default


%%% Castellano.
% noquoting: Permite uso de comillas no españolas.
% lcroman: Permite la enumeración con numerales romanos en minúscula.
% fontenc: Usa la fuente completa para que pueda copiarse correctamente del pdf.
\usepackage[spanish,es-noquoting,es-lcroman]{babel}
\usepackage[utf8]{inputenc}
\usepackage[T1]{fontenc}
\selectlanguage{spanish}


%%% Gráficos
\usepackage{graphicx} % Required for including pictures
\usepackage{wrapfig} % Allows in-line images
\usepackage[usenames,dvipsnames]{color} % Coloring code


%%% Matemáticas
\usepackage{amsmath}
\usepackage[hidelinks]{hyperref}
%%% Código


\usepackage{listings}
\usepackage{graphicx}

%% Listing settings

\usepackage{color}

\definecolor{dkgreen}{rgb}{0,0.6,0}
\definecolor{gray}{rgb}{0.5,0.5,0.5}
\definecolor{mauve}{rgb}{0.58,0,0.82}

\lstset{frame=tb,
  language=Java,
  aboveskip=3mm,
  belowskip=3mm,
  showstringspaces=false,
  columns=flexible,
  basicstyle={\small\ttfamily},
  numbers=none,
  numberstyle=\tiny\color{gray},
  keywordstyle=\color{blue},
  commentstyle=\color{dkgreen},
  stringstyle=\color{mauve},
  breaklines=true,
  breakatwhitespace=true,
  tabsize=3
}



%%% Bibliografía
\makeatletter
\renewcommand\@biblabel[1]{\textbf{#1.}} % Change the square brackets for each bibliography item from '[1]' to '1.'
\renewcommand{\@listI}{\itemsep=0pt} % Reduce the space between items in the itemize and enumerate environments and the bibliography



%----------------------------------------------------------------------------------------
%	TÍTULO
%----------------------------------------------------------------------------------------
% Configuraciones para el título.
% El título no debe editarse aquí.
\renewcommand{\maketitle}{
  \begin{flushright} % Right align
  
  {\LARGE\@title} % Increase the font size of the title
  
  \vspace{50pt} % Some vertical space between the title and author name
  
  {\large\@author} % Author name
  \\\@date % Date
  \vspace{40pt} % Some vertical space between the author block and abstract
  \end{flushright}
}

%% Título
\title{\textbf{Memoria de la práctica 3}\\ % Title
Algorítmica} % Subtitle

\author{\textsc{Fco. Javier Sáez Maldonado}\\ % Author
\textsc{Laura Gómez Garrido}\\
\textsc{Luis Antonio Ortega Andrés}\\
\textsc{Pedro Bonilla Nadal}\\
\textsc{Daniel Pozo Escalona}\vspace{2cm}
\\{\textit{Universidad de Granada}}} % Institution

\date{\today} % Date



%----------------------------------------------------------------------------------------
%	DOCUMENTO
%----------------------------------------------------------------------------------------

\begin{document}

\maketitle % Print the title section


%% Índice
{\parskip=2pt
  \tableofcontents
}
\pagebreak

%%% Inicio del documento


\section{Introducción}

Un polígono $P$ en el plano euclídeo es un conjunto de $n$ puntos $(x_i, y_i) \in \mathbb{R}^2,\ i=1,\dots,n$ llamados vértices, y $n$ segmentos de rectas ${\alpha_1 , \alpha_2 ,\dots, \alpha_n }$ llamados lados, que verifican que:

\begin{itemize}
	\item Los puntos extremos de los lados son dos vértices distintos del polígono. 
	\item Cualquier vértice es el extremo de exactamente dos lados (distintos).
\end{itemize}

Un polígono del plano se dice \textbf{convexo} si cada uno de sus vértices es un vértice de su envolvente convexa.

 Por \textbf{triangulación} de un polígono plano entenderemos la división de este en un conjunto de triángulos, con la restricción de que dos lados de dos triángulos distintos no se corten o lo hagan en un vértice del polígono.

Diremos que una triangulación de un polígono convexo es \textbf{mínima} si la suma de los perímetros de los triángulos que la componen es mínima respecto de cualquier otra triangulación.

\section{Problema}

Se pide diseñar e implementar un algoritmo \textit{greedy} que solucione el problema de encontrar la triangulación mínima de un polígono convexo. Como salida, se deberá proporcionar el conjunto de aristas que forman la triangulación junto con la suma de los perímetros de los triángulos.

\section{Solución}

\subsection{Descripción del algoritmo}
La entrada del algoritmo es un conjunto de $n$ puntos del plano euclídeo, los vértices del polígono convexo que queremos triangular, ordenados. De esta forma, los lados del polígono quedan determinados por el orden, es decir, son: 
\[\alpha_i \vee \alpha_{i+1} \forall i\in \mathbb{Z}_n\]

Del mismo modo, las cuerdas son:
\[\alpha_i \vee \alpha_{i+2} \forall i\in \mathbb{Z}_n\]

Los pasos del algoritmo:

\begin{enumerate}
    \item Si el polígono es un triángulo, no se hace nada.
	\item Encontrar la cuerda de longitud mínima.
	\item Almacenar la cuerda en el conjunto solución.
	\item Eliminar del conjunto de vértices el comprendido entre los dos vértices que une la cuerda.
	\item Ejecutar el algoritmo desde el primer paso, siendo esta vez la entrada el conjunto de vértices resultante en el paso anterior.
\end{enumerate}

\subsection{Implementación}
Para la implementación, hemos escogido el lenguaje de programación \texttt{C++}.

\subsubsection{Estructuras de datos}
Para almacenar los vértices del polígono, hemos escogido el contenedor \texttt{list} de la \textit{STL} de \texttt{C++}. Además, hemos definido una estructura para almacenar los puntos, con un par de coordenadas en punto flotante de doble precisión y un identificador numérico para la salida por pantalla.

\subsubsection{Avance circular de un iterador}
\begin{lstlisting}

auto circular_advance = [](auto& forwdIt, auto initValue, auto endValue, int n) {
	for(int i=0; i<n; i++)
	{
		forwdIt++;
		if(forwdIt == endValue)
		forwdIt = initValue;
	}

	return forwdIt;
};
\end{lstlisting}

Esta es una función auxiliar, cuyo propósito es implementar la aritmética de un anillo de restos sobre un contenedor con un iterador que soporte el operador de incremento (\texttt{++}).

La eficiencia teórica de esta función es $O(n)$, sin embargo, siempre se llama con un argumento constante.

\subsubsection{Cuerda de longitud mínima}
\begin{lstlisting}
	
auto find_min_string = [&min_distance, &it_min, &min_string, &points](auto initial_point_it) {
	auto p0 = initial_point_it;
	auto p = p0;
	
	do {
		auto prev = p;
	
		circular_advance(p, points.begin(), points.end(), 2);
		if(min_distance > euclidean_distance(*prev, *p))
		{
			min_distance = euclidean_distance(*prev, *p);
			min_string = std::make_pair(*prev, *p);
			it_min = circular_advance(prev, points.begin(), points.end(), 1);
		}
	
	}while(p != p0);
};
\end{lstlisting}

Esta función acumula la longitud de la cuerda mínima, empezando en un vértice y recorriendo las cuerdas hasta volver al primero. Si el número de vértices del polígono es impar, solo tiene un ciclo de cuerdas, en cambio tiene dos si es par ($\mathbb{Z}_n/2\mathbb{Z}_n \simeq \mathbb{F}_1$ si $n$ es impar, $\simeq \mathbb{F}_2$ si $n$ es par). Por tanto, en un polígono cuyo número de vértices sea par, ha de llamarse dos veces.

La eficiencia teórica de esta función es $O(n)$.
\subsubsection{Bucle principal}

	\begin{lstlisting}
	while(points.size() > 3)
	{
	auto it = points.begin();
	// Unconditionally find minimum length string starting
	// at the first element
	find_min_string(it);
	
	// If the polygon's number of edges is even, we need to
	// do the same starting at the second one
	if(!(points.size()%2)) find_min_string(++it);
	
	sol.push_back(min_string);
	points.erase(it_min);
	}
	\end{lstlisting}
	
	Dado un polígono de $n$ nodos, $n$ puede ser par o no serlo, en el caso de que no sea par, la función \textit{find\_min\_string} recorrera todos los nodos sin problemas. Sin embargo, en el caso de que sea par, la función solo recorrera la mitad de los nodos.
	Por ello, se vuelve a llamar a la función para recorrer los nodos faltantes.

\subsubsection{La complejidad del algoritmo}
Es un algoritmo de complejidad $O(n^2)$, dado que para un polígono de $n$ nodos, en cada iteración del bucle while ($n$ iteraciones) realiza la función \textit{find\_min\_string} la cual realiza $n$ iteraciones en las que en cada una de ellas llama a \textit{circular\_advance} con $2$ como parámetro ($O(2)$).


\section{Conclusión}



\end{document}
