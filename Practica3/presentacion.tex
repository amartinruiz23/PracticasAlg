% Email: pbaeyens31+github@gmail.com
% Licencia: CC BY-SA 3.0

%% Paquetes y configuración %

% Beamer
\PassOptionsToPackage{unicode}{hyperref}  % Evita errores con caracteres no ASCII
\PassOptionsToPackage{naturalnames}{hyperref} % tex.stackexchange.com/questions/10555
\documentclass[compress]{beamer}
\usepackage{adjustbox}
% Idioma
\usepackage[spanish]{babel} % Traducciones
\usepackage[utf8]{inputenc} % Uso de caracteres UTF-8
\usepackage{lmodern}        % Fuentes de tamaño arbitrario
\usepackage[T1]{fontenc}    % Permite copiar y evita errores
%\uselanguage{Spanish}       % Traducciones beamer
%\languagepath{Spanish}      % (tex.stackexchange.com/questions/168208)

\usepackage{multicol}
\usepackage{multirow}

% Matemáticas
\usepackage{amsfonts}
\usepackage{amsmath}
\usepackage{amssymb}
\usepackage{marginnote}
% Colores
\definecolor{backg}{HTML}{F2F2F2}    % Fondo
\definecolor{title}{HTML}{bdc3d1}    % Títulos
\definecolor{comments}{HTML}{BDBDBD} % Comentarios
\definecolor{keywords}{HTML}{08388c} % Palabras clave
\definecolor{strings}{HTML}{FA5858}  % Strings
\definecolor{links}{HTML}{2C2C95}    % Enlaces
\definecolor{bars}{HTML}{045FB4}     % Barras (gráfico)

% Código
\usepackage{listings}
\lstset{
language=[LaTeX]TeX,
basicstyle=\footnotesize,
morekeywords={href,uselanguage,languagepath,column},
otherkeywords={pause,usetheme,usecolortheme,useinnertheme,titlepage,tableofcontents,subtitle},
breaklines=true,
backgroundcolor=\color{backg},
keywordstyle=\color{keywords},
commentstyle=\color{comments},
stringstyle=\color{strings},
tabsize=2,
% Acentos, ñ, ¿, ¡ (tex.stackexchange.com/questions/24528)
extendedchars=true,
literate={á}{{\'a}}1 {é}{{\'e}}1 {í}{{\'i}}1 {ó}{{\'o}}1
         {ú}{{\'u}}1 {ñ}{{\~n}}1 {¡}{{\textexclamdown}}1
         {¿}{{?`}}1
}

% Gráficos
\usepackage{pgfplots}
\pgfplotsset{width=7cm,compat=1.14} % Opciones para gráficos

% Emoticonos
\usepackage{wasysym}
\usepackage{graphicx}
% tikz
\usepackage{tikz}
\usetikzlibrary{mindmap,trees,shadows}
\tikzset{ % Genera overlays
    invisible/.style={opacity=0},
    visible on/.style={alt={#1{}{invisible}}},
    alt/.code args={<#1>#2#3}{\alt<#1>{\pgfkeysalso{#2}}{\pgfkeysalso{#3}}},
}

%% Comandos %%
\newcommand{\ejemplo}[1]{\lstinputlisting{./examples/#1}} % Mostrar código de ejemplos
\newcommand{\muestra}[1]{\input{./examples/#1}}           % Mostrar ejemplos
\newcommand{\seccion}[1]{\input{./sections/#1}}           % Incluir secciones
\newcommand{\espacio}{\vspace*{\baselineskip}}            % Añade espacios
\newcommand{\beamer}{\texttt{beamer} }                    % Estilo único para beamer
\newcommand{\enlace}[3]{\href{#1}{\textbf{#2}} - {\small #3}}  % Estílo único para refs
\newcommand{\comando}[1]{{\color{black}\textbackslash}{\color{keywords}#1}}
\newcommand{\marginalnote}[1]{\mbox{}\marginpar{\raggedright\hspace{0pt}#1}}
%% Temas %%
% Tema y tema de color
  \usetheme{Dresden}
  \usecolortheme{dolphin}
  \useinnertheme{circles}
  \setbeamercovered{transparent}
% Colores bloques
  \setbeamercolor{block title}{bg=title,fg=links}
  \setbeamercolor{block body}{bg=backg,fg=black}
  \setbeamercolor{block title alerted}{fg=red!70!black,bg=title!92!red}
  \setbeamercolor{block body alerted}{fg=black,bg=backg}
  \setbeamercolor{block title example}{fg=green!70!black,bg=title!92!green}
  \setbeamercolor{block body example}{fg=black,bg=backg}
% Enlaces (tex.stackexchange.com/questions/13423)
\hypersetup{colorlinks,linkcolor=,urlcolor=links}
% Quita enlaces de navegación (stackoverflow.com/questions/3017030)
\setbeamertemplate{navigation symbols}{}
% Quita barra inferior (stackoverflow.com/questions/1435837)
\setbeamertemplate{footline}{}
% Evita warnings boxes
\hfuzz=20pt
\vfuzz=20pt
% Evita wranings itemize
\renewcommand\textbullet{\ensuremath{\bullet}}

%% Título y otros %%
\title{Práctica 3}                                               % Título
\subtitle{Greedy}                                  % Subtítulo
\author{Laura Gómez Garrido, Pedro Bonilla Nadal, Javier Sáez Maldonado, Daniel Pozo Escalona, Luis Ortega Andrés}

\date{2º Doble Grado Informática y Matemáticas}                                                            % Fecha

%% Listing settings

\usepackage{color}

\definecolor{dkgreen}{rgb}{0,0.6,0}
\definecolor{gray}{rgb}{0.5,0.5,0.5}
\definecolor{mauve}{rgb}{0.58,0,0.82}

\lstset{frame=tb,
  language=Java,
  aboveskip=3mm,
  belowskip=3mm,
  showstringspaces=false,
  columns=flexible,
  basicstyle={\small\ttfamily},
  numbers=none,
  numberstyle=\tiny\color{gray},
  keywordstyle=\color{blue},
  commentstyle=\color{dkgreen},
  stringstyle=\color{mauve},
  breaklines=true,
  breakatwhitespace=true,
  tabsize=3
}

%% Presentación %%
\begin{document}

\begin{frame}
\titlepage
\end{frame}

\begin{frame}{Índice}
  \hypertarget{index}{}
  \tableofcontents
  
\end{frame}

\section{Problema}
\begin{frame}

\textbf{\\Problema.\\}
	Se pide diseñar e implementar un algoritmo greedy que solucione el problema de encontrar la triangulación mínima de un polígono convexo. Como salida, se deberá proporcionar el conjunto de aristas que forman la triangulación junto con el coste (perímetro de cada triángulo generado).

\end{frame}	

\section{Poligono convexo}
\begin{frame}
	
	\textbf{\\Conceptos.\\}
Un polígono del plano se dice \textbf{convexo} si, para cada dos puntos cualesquiera del interior del polígono, el segmento que los une está contenido en el mismo. \\
Por la \textbf{triangulación de un polígono plano} entenderemos la división de éste en un conjunto de triángulos, con la restricción de que cada lado de un triángulo se reparta entre dos triángulos adyacentes (obviamente, sólo pertenece a uno si está en la frontera del polígono).
	
	TODO: EJEMPLO POLIGONO TRIANGULADO
\end{frame}	
	
\section{Solución}

\subsection{Explicación Algoritmo}

\begin{frame}
	
	\textbf{\\Explicación Algoritmo.\\}
\begin{itemize}
	\item Obtener todas las cuerdas (segmentos que unen nodos que se encuentran a 2 de distancia) del polígono.
	\item Seleccionar la cuerda de menor longitud.
	\item Eliminar el nodo aislado por dicha cuerda.
	\item Aplicar el algoritmo al polígono resultante.
	
	\item El caso base se cumple cuando el poligono tiene 4 nodos.
\end{itemize}

\end{frame}	

\begin{frame}[fragile]
	\frametitle{Circular Advance}
 \begin{lstlisting}
auto circular_advance = [](auto& forwdIt, auto initValue, auto endValue, int n) {
	for(int i=0; i<n; i++)
	{
		forwdIt++;
		if(forwdIt == endValue)
			forwdIt = initValue;
	}

	return forwdIt;
};
 \end{lstlisting}
 
 Avanza un elemento (iterator or integer) como en una lista circular de n en n.
 
\end{frame}

\begin{frame}[fragile]
	\frametitle{Find Min String}
\begin{lstlisting}[basicstyle= \tiny,]
    auto find_min_string = [&min_distance, &it_min, &min_string, &points](auto initial_point_it) {
	    auto p0 = initial_point_it;
	    auto p = p0;
    
	    do {
		   auto prev = p;
    
		   circular_advance(p, points.begin(), points.end(), 2);
		   if(min_distance > euclidean_distance(*prev, *p))
		   {
		      min_distance = euclidean_distance(*prev, *p);
              min_string = std::make_pair(*prev, *p);
              it_min = circular_advance(prev, points.begin(), points.end(), 1);
            }
    
        }while(p != p0);
    };
\end{lstlisting}
	Busca la cuerda de mínima longitud recorriendo los nodos hasta volver al nodo inicial.
\end{frame}

\begin{frame}[fragile]
	\frametitle{Algoritmo}
	\begin{lstlisting}
	while(points.size() > 3)
	{
		auto it = points.begin();
		// Unconditionally find minimum length string starting
		// at the first element
		find_min_string(it);
	
		// If the polygon's number of edges is even, we need to
		// do the same starting at the second one
		if(!(points.size()%2)) find_min_string(++it);
	
		sol.push_back(min_string);
		points.erase(it_min);
	}
	\end{lstlisting}
	
\end{frame}

\end{document}
