

% microtype: Tipografía.
% mathpazo: Usa la fuente Palatino.
\documentclass[a4paper, 11pt]{article}
\usepackage[protrusion=true,expansion=true]{microtype}
\usepackage{mathpazo}

% Indentación de párrafos para Palatino
\setlength{\parindent}{0pt}
  \parskip=8pt
\linespread{1.05} % Change line spacing here, Palatino benefits from a slight increase by default


%%% Castellano.
% noquoting: Permite uso de comillas no españolas.
% lcroman: Permite la enumeración con numerales romanos en minúscula.
% fontenc: Usa la fuente completa para que pueda copiarse correctamente del pdf.
\usepackage[spanish,es-noquoting,es-lcroman]{babel}
\usepackage[utf8]{inputenc}
\usepackage[T1]{fontenc}
\selectlanguage{spanish}


%%% Gráficos
\usepackage{graphicx} % Required for including pictures
\usepackage{wrapfig} % Allows in-line images
\usepackage[usenames,dvipsnames]{color} % Coloring code


%%% Matemáticas
\usepackage{amsmath}


%%% Bibliografía
\makeatletter
\renewcommand\@biblabel[1]{\textbf{#1.}} % Change the square brackets for each bibliography item from '[1]' to '1.'
\renewcommand{\@listI}{\itemsep=0pt} % Reduce the space between items in the itemize and enumerate environments and the bibliography



%----------------------------------------------------------------------------------------
%	TÍTULO
%----------------------------------------------------------------------------------------
% Configuraciones para el título.
% El título no debe editarse aquí.
\renewcommand{\maketitle}{
  \begin{flushright} % Right align
  
  {\LARGE\@title} % Increase the font size of the title
  
  \vspace{50pt} % Some vertical space between the title and author name
  
  {\large\@author} % Author name
  \\\@date % Date
  \vspace{40pt} % Some vertical space between the author block and abstract
  \end{flushright}
}

%% Título
\title{\textbf{Práctica 1}\\ % Title
Algorítmica} % Subtitle

\author{\textsc{Fco. Javier Sáez Maldonado}\\ % Author
\textsc{Laura Gómez Garrido}\\
\textsc{Luis Antonio Ortega Andrés}\\
\textsc{Pedro Bonilla Nadal}\\
\textsc{Daniel Pozo Escalona}\vspace{2cm}
\\{\textit{Universidad de Granada}}} % Institution

\date{\today} % Date



%----------------------------------------------------------------------------------------
%	DOCUMENTO
%----------------------------------------------------------------------------------------

\begin{document}

\maketitle % Print the title section


%% Índice
{\parskip=2pt
  \tableofcontents
}
\pagebreak

%%% Inicio del documento


\section*{Introducción}

En esta primera práctica, vamos a centrarnos en el estudio de la eficiencia tanto empírica como teórica de ciertos algoritmos. Para ello, realizaremos pruebas empíricas con nuestros propios equipos para comprobar que la eficiencia empírica se ajusta de forma más o menos adecuada a la eficiencia que calcularemos de forma teórica.

Para ello, comenzaremos presentando los algoritmos que vamos a estudiar. Son algoritmos bastante conocidos y cuyas eficiencias teóricas también son conocidas. Estos son:
\begin{enumerate}
	\item Algoritmo de ordenación \textbf{burbuja}
	\item Algoritmo de ordenación por \textbf{inserción}
	\item Algoritmo de ordenación por \textbf{selección}
	\item Algoritmo de ordenación \textbf{mergesort}, basado en la técnica divide y vencerás
	\item Algoritmo de ordenación \textbf{quicksort}
	\item Algoritmo de ordenación \textbf{heapsort}
	\item Algoritmo de Floyd
	\item Algoritmo de Hanoi
\end{enumerate}

Para obtener unos buenos resultados, hemos decidido hacer las ejecuciones de estos algoritmos en varias máquinas. Las máquinas que usaremos serán:
\begin{enumerate}    
	\item Máquina A: Procesador Intel Core I7-5700HQ, 6M Cache y 3.5 Ghz.
	\item Máquina B: Procesador: Intel Core i7-4510U ,2.00GHz
	\item Máquina C: Máquina Virtual VirtualBox version 5.1.112r112440(Qt5.6.2) dentro de una Máquina B.

\end{enumerate}

\section{Cálculo de la eficiencia empírica}


\end{document}